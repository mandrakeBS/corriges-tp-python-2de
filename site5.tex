% ============================================================
%  site5.tex — Version stable (mode clair/sombre fonctionnel)
% ============================================================

\makeatletter
\@ifundefined{HCode}{\newcommand{\HCode}[1]{}}{}
\makeatother

\documentclass[12pt,a4paper]{article}
% --- Encodage & langue ---
\usepackage[utf8]{inputenc}
\usepackage[T1]{fontenc}
\usepackage[french]{babel}

% --- Polices sûres ---
\usepackage{lmodern}
\usepackage{helvet}
\renewcommand{\familydefault}{\sfdefault}

% --- Math & symboles ---
\usepackage{amsmath,amssymb,systeme}

% --- Tableaux & couleurs ---
\usepackage{booktabs,multirow,colortbl,tabularx,xcolor}

% --- Listing ---
%\usepackage{xcolor} %déjà défini
\usepackage{listings}
\lstset{
literate=
{á}{{\'a}}1 {é}{{\'e}}1 {í}{{\'i}}1 {ó}{{\'o}}1 {ú}{{\'u}}1
{Á}{{\'A}}1 {É}{{\'E}}1 {Í}{{\'I}}1 {Ó}{{\'O}}1 {Ú}{{\'U}}1
{à}{{\`a}}1 {è}{{\`e}}1 {ì}{{\`i}}1 {ò}{{\`o}}1 {ù}{{\`u}}1
{À}{{\`A}}1 {È}{{\'E}}1 {Ì}{{\`I}}1 {Ò}{{\`O}}1 {Ù}{{\`U}}1
{ä}{{\"a}}1 {ë}{{\"e}}1 {ï}{{\"i}}1 {ö}{{\"o}}1 {ü}{{\"u}}1
{Ä}{{\"A}}1 {Ë}{{\"E}}1 {Ï}{{\"I}}1 {Ö}{{\"O}}1 {Ü}{{\"U}}1
{â}{{\^a}}1 {ê}{{\^e}}1 {î}{{\^i}}1 {ô}{{\^o}}1 {û}{{\^u}}1
{Â}{{\^A}}1 {Ê}{{\^E}}1 {Î}{{\^I}}1 {Ô}{{\^O}}1 {Û}{{\^U}}1
{œ}{{\oe}}1 {Œ}{{\OE}}1 {æ}{{\ae}}1 {Æ}{{\AE}}1 {ß}{{\ss}}1
{ű}{{\H{u}}}1 {Ű}{{\H{U}}}1 {ő}{{\H{o}}}1 {Ő}{{\H{O}}}1
{ç}{{\c c}}1 {Ç}{{\c C}}1 {ø}{{\o}}1 {å}{{\r a}}1 {Å}{{\r A}}1
{€}{{\EUR}}1 {£}{{\pounds}}1
}
\definecolor{pyrouge}{rgb}{0.77, 0.12, 0.23}
\definecolor{pyvert}{rgb}{0.0, 0.5, 0.0}
%========================================
\lstdefinestyle{monstylepython1}{
    language = Python,
    basicstyle=\ttfamily,
    backgroundcolor=\color{yellow!10},
    stringstyle = {\color{string-color}},
    keywordstyle = \color{pyvert},%blue
    commentstyle=\color{orange},
    keywordstyle = [2]{\color{blue}},
    stringstyle=\color{pyrouge},%olive
    showstringspaces=false,%true sans cette ligne
    otherkeywords = {},
    %morekeywords = [2]{foncAffine_g, estSurDg},
    showstringspaces=false,%true sans cette ligne
    numberstyle=\tiny,  
    numbers=left,
    stepnumber=1, 
    numbersep=5pt,
    %frame=single 
    }
    
    \lstdefinestyle{monstylepython2}{
    language = Python,
    basicstyle=\ttfamily,
    backgroundcolor=\color{blue!10},
    stringstyle = {\color{string-color}},
    keywordstyle = \color{pyvert},%blue
    commentstyle=\color{orange},
    keywordstyle = [2]{\color{blue}},
    stringstyle=\color{pyrouge},%olive
    showstringspaces=false,%true sans cette ligne
    otherkeywords = {},
    %morekeywords = [2]{foncAffine_g, estSurDg},
    showstringspaces=false,%true sans cette ligne
    numberstyle=\tiny,  
    numbers=left,
    stepnumber=1, 
    numbersep=5pt,
    frame=single 
    }
	\lstdefinestyle{monstylepython3}{
	language = Python,
	basicstyle=\ttfamily\small,
  	backgroundcolor=\color{yellow!10},
  	frame=single,
  	numbers=left,
  	numberstyle=\tiny,
  	stepnumber=1,
  	showstringspaces=false,
  	keywordstyle=\color{blue},
  	commentstyle=\color{gray!80!black},
  	stringstyle=\color{red!60!black}
	}	
	
% --- Graphiques (TikZ) ---
\usepackage{tikz}
\usetikzlibrary{arrows,patterns}

% --- PDF uniquement ---
\ifdefined\HCode
  % rien pour HTML
\else
  \usepackage{geometry}
  \geometry{margin=2cm}
  \usepackage{fancyhdr}
  \pagestyle{fancy}
  \fancyhf{}
  \fancyhead[L]{\leftmark}
  \fancyfoot[C]{\thepage}
  \usepackage{hyperref}
  \usepackage{qrcode}
\fi

\newcommand{\HTML}[1]{%
  \begingroup
    \catcode`\<=12 % < ordinaire
    \catcode`\>=12 % > ordinaire
    \catcode`\!=12 % ! ordinaire
    \catcode`\%=12 % % ordinaire
    \catcode`\'=12 % apostrophe simple
    \catcode`\"=12 % guillemet double
    \HCode{#1}%
  \endgroup}

\makeatletter
% Balise HTML : <tag> ... </tag> en HTML ; fallback lisible en PDF
\newcommand{\Htag}[2]{%
  \ifdefined\HCode
    \HCode{<#1>}#2\HCode{</#1>}%
  \else
    \par\textbf{#2}\par
  \fi
}

% <span class="..."> ... </span> en HTML ; fallback neutre en PDF
\newcommand{\Hclass}[2]{%
  \ifdefined\HCode
    \HCode{<span class='#1'>}#2\HCode{</span>}%
  \else
    #2%
  \fi
}
\makeatother


% --- Métadonnées ---
\title{Travaux Pratiques – Python}
\author{Bruno S.}
\date{\today}

\begin{document}
\maketitle

% === Lien CSS et JS ===
\HCode{<link rel='stylesheet' type='text/css' href='style.css'>}
\HCode{<script src='script.js' defer></script>}
\HCode{<button id='theme-toggle' onclick='toggleMode()'>☀️</button>}

\HCode{<h1>Corrigés des Travaux Pratiques Python</h1>}
\HCode{<p>Ces exercices peuvent être exécutés directement dans votre navigateur.
Cliquez sur <strong>Run</strong> dans les fenêtres ci-dessous.</p>}
%================
%!     TP01     !
%================
\HCode{<div class='exercice'>}
\HCode{<h2>TP01 – Déterminer si \(a\) est multiple de \(b\).</h2>}
%\HCode{<p>\(a,b\in N\)</p>}
\Htag{p}{\(a,b\in \mathbb{N}\)}
%\HCode{<iframe src='https://trinket.io/embed/python3/123abc456'></iframe>}
\HCode{<button data-target="corr1">Afficher / masquer une solution</button>}
\HTML{<div id='corr1' class='correction'>}
\HTML{<ol>}
\HTML{<li>Des quatre résultats obtenus, on peut en déduire qu'il s'agit du reste de la division euclidienne de \(a\) par \(b\) (réponse b).</li>}
%\HTML{<li>Les six opérations sont: </li>}
\Htag{li}{Les six opérations sont: ==; !=, <; <=; >; >=}
\HTML{<li>}
\begin{lstlisting}[language=Python, style=monstylepython2, firstnumber=15]
if a%b == 0:
\end{lstlisting}
\HTML{</li>}
\HTML{</ol></div></div>}

%================
%!     TP02     !
%================
\HCode{<div class='exercice'>}
\HCode{<h2>TP02 - Déterminer le plus grand multiple de \(a\) inférieur ou égal à \(b\)</h2>}
\HCode{<p>-</p>}
%\HCode{<iframe src='https://trinket.io/embed/python3/789xyz123'></iframe>}
\HCode{<button data-target="corr2">Afficher / masquer une solution</button>}
\HCode{<div id='corr2' class='correction'>}
\lstinputlisting[language=Python, style=monstylepython3]{Python_et_notebook/TP02_corr.py}
\HCode{</div></div>}

%================
%!     TP03     !
%================
\HCode{<div class='exercice'>}
\HCode{<h2>TP03 - Déterminer si un entier naturel est premier (calcul brut)</h2>}
\HCode{<p>-</p>}
%\HCode{<iframe src='https://trinket.io/embed/python3/789xyz123'></iframe>}
\HCode{<button data-target="corr3">Afficher / masquer une solution</button>}
\HCode{<div id='corr3' class='correction'><pre>}
\lstinputlisting[language=Python, style=monstylepython1]{Python_et_notebook/TP03_corr.py}
\HCode{</pre></div></div>}

%================
%!     TP04     !
%================
\HCode{<div class='exercice'>}
\HCode{<h2>TP04 - Déterminer si un entier naturel est premier (exploitation du "critère de primalité")</h2>}
\HCode{<p>-</p>}
\HCode{<iframe src='https://trinket.io/embed/python3/789xyz123'></iframe>}
\HCode{<button data-target="corr4">Afficher / masquer une solution</button>}
\HCode{<div id='corr4' class='correction'><pre>}
\lstinputlisting[language=Python, style=monstylepython1]{Python_et_notebook/TP04_corr.py}
\HCode{</pre></div></div>}

%================
%!     TP05     !
%================
\HCode{<div class='exercice'>}
\HCode{<h2>TP05 - Déterminer la première puissance d'un nombre positif donné supérieure ou inférieure à une valeur donnée</h2>}
\HCode{<p>-</p>}
\HCode{<iframe src='https://trinket.io/embed/python3/789xyz123'></iframe>}
\HCode{<button data-target="corr5">Afficher / masquer une solution</button>}
\HCode{<div id='corr5' class='correction'><pre>}
\lstinputlisting[language=Python, style=monstylepython1]{Python_et_notebook/TP05_corr.py}
\HCode{</pre></div></div>}

%================
%!     TP06     !
%================
\HCode{<div class='exercice'>}
\HCode{<h2>TP06 - Approximation numérique d'un extremum d'une fonction (par balayage)</h2>}
\HCode{<p>-</p>}
\HCode{<iframe src='https://trinket.io/embed/python3/789xyz123'></iframe>}
\HCode{<button data-target="corr6">Afficher / masquer une solution</button>}
\HCode{<div id='corr6' class='correction'><pre>}
\lstinputlisting[language=Python, style=monstylepython1]{Python_et_notebook/TP06_corr.py}
\HCode{</pre></div></div>}

%================
%!     TP07     !
%================
\HCode{<div class='exercice'>}
\HCode{<h2>TP07 - Approximation numérique d'un extremum d'une fonction (par dichotomie)</h2>}
\HCode{<p>-</p>}
\HCode{<iframe src='https://trinket.io/embed/python3/789xyz123'></iframe>}
\HCode{<button data-target="corr7">Afficher / masquer une solution</button>}
\HCode{<div id='corr7' class='correction'><pre>}
\lstinputlisting[language=Python, style=monstylepython1]{Python_et_notebook/TP07_corr.py}
\HCode{</pre></div></div>}

%================
%!     TP08     !
%================
\HCode{<div class='exercice'>}
\Htag{h2}{TP08 - Détermination d'un encadrement de $ \sqrt{2} $
d'amplitude $ d\leqslant 10^{-n} $ (par balayage)}
\HCode{<p>-</p>}
\HCode{<iframe src='https://trinket.io/embed/python3/789xyz123'></iframe>}
\HCode{<button data-target="corr8">Afficher / masquer une solution</button>}
\HCode{<div id='corr8' class='correction'><pre>}
\lstinputlisting[language=Python, style=monstylepython1]{Python_et_notebook/TP08_corr.py}
\HCode{</pre></div></div>}

%================
%!     TP09     !
%================
\HCode{<div class='exercice'>}
\Htag{h2}{TP09 - Détermination d'un encadrement de $ \sqrt{2} $
d'amplitude $ d\leqslant 10^{-n} $ (par dichotomie)}
\HCode{<p>-</p>}
\HCode{<iframe src='https://trinket.io/embed/python3/789xyz123'></iframe>}
\HCode{<button data-target="corr9">Afficher / masquer une solution</button>}
\HCode{<div id='corr9' class='correction'><pre>}
\lstinputlisting[language=Python, style=monstylepython1]{Python_et_notebook/TP09_corr.py}
\HCode{</pre></div></div>}

%================
%!     TP10     !
%================
\HCode{<div class='exercice'>}
\HCode{<h2>TP10 - Étudier l'alignement de trois points dans le plan (en calculant l'aire d'un triangle)</h2>}
\HCode{<p>-</p>}
\HCode{<iframe src='https://trinket.io/embed/python3/789xyz123'></iframe>}
\HCode{<button data-target="corr10">Afficher / masquer une solution</button>}
\HCode{<div id='corr10' class='correction'><pre>}
\lstinputlisting[language=Python, style=monstylepython1]{Python_et_notebook/TP10_corr.py}
\HCode{</pre></div></div>}

%================
%!     TP11     !
%================
\HCode{<div class='exercice'>}
\HCode{<h2>TP11 - Étudier l'alignement de trois points dans le plan (en calculant un déterminant)</h2>}
\HCode{<p>-</p>}
\HCode{<iframe src='https://trinket.io/embed/python3/789xyz123'></iframe>}
\HCode{<button data-target="corr11">Afficher / masquer une solution</button>}
\HCode{<div id='corr11' class='correction'><pre>}
\lstinputlisting[language=Python, style=monstylepython1]{Python_et_notebook/TP11_corr.py}
\HCode{</pre></div></div>}

%================
%!     TP12     !
%================
\HCode{<div class='exercice'>}
\HCode{<h2>TP12 - Résolution d'un système de deux équations du premier degré à deux inconnues)</h2>}
\HCode{<p>-</p>}
\HCode{<iframe src='https://trinket.io/embed/python3/789xyz123'></iframe>}
\HCode{<button data-target="corr12">Afficher / masquer une solution</button>}
\HCode{<div id='corr12' class='correction'><pre>}
%\lstinputlisting[language=Python, style=monstylepython1]{Python_et_notebook/TP12_corr.py}
\HCode{</pre></div></div>}

\HCode{<footer>© 2025 – Corrigés et programmes créés par Bruno S.</footer>}

\end{document}